\documentclass{llncs}
\usepackage{epsfig}
% \usepackage{color}
\usepackage{alltt}
\usepackage{ulem}
\normalem

\begin{document}

\newcommand{\name}[1]{{\sf\textsl{#1}}}        % objects/processes
\newcommand{\vat}[1]{{\sf Vat{#1}}}            % vats
\newcommand{\pr}[1]{{#1}}                      % principals
\newcommand{\code}[1]{{\tt {#1}}}              % code
\newcommand{\var}[1]{{\tt {#1}}}               % variables
\newcommand{\dvar}[1]{{\textsl{#1}}}           % declarations of variables
\newcommand{\dobj}[1]{{\textsl{#1}}}           % declarations of objects
\newcommand{\meth}[1]{{\tt {#1}}}              % methods
\newcommand{\dmeth}[1]{{\tt {#1}}}             % declarations of methods
\newcommand{\cls}[1]{{\tt {#1}}}               % classes
\newcommand{\ex}[1]{{\tt {#1}}}                % exceptions
\newcommand{\abst}[1]{{#1}}                    % design abstractions
% \newcommand{\sys}[1]{{sc {#1}}}              % systems and languages
\newcommand{\sys}[1]{{#1}}                     % systems and languages

% \newcommand{\related}[1]{\subsubsection{{#1}}}
\newcommand{\related}[1]{\vspace{3pt}\textbf{{#1}}}

\def\twiddle{\raisebox{0.3ex}{\mbox{\tiny $\sim$}}}

\hyphenation{status-Holder}

\title{Concurrency Among Strangers}
\subtitle{Programming in E as Plan Coordination}

\author{Mark S. Miller\inst{1,2} \and 
  E. Dean Tribble \and
  Jonathan Shapiro\inst{1}}

\institute{Johns Hopkins University
\and Hewlett Packard Laboratories}

\maketitle

\sloppypar

\begin{abstract}
Programmers write programs, expressing plans for machines to
execute. When composed so that they may cooperate, plans may instead
interfere with each other in unanticipated ways. \emph{Plan
coordination} is the art of simultaneously enabling plans to
cooperate, while avoiding hazards of destructive plan
interference. For sequential computation within a single machine,
object programming supports plan coordination well. For concurrent
computation, this paper shows how hard it is to use locking to prevent
plans from interfering without also destroying their ability to
cooperate.\\

In Internet-scale computing, machines proceed concurrently, interact
across barriers of large latencies and partial failure, and encounter
each other's misbehavior. Each dimension presents new plan
coordination challenges. This paper explains how the E language
addresses these joint challenges by changing only a few concepts of
conventional sequential object programming. Several projects are
adapting these insights to existing platforms.
\end{abstract}

\section{Introduction}

The fundamental constraint we face as programmers is complexity. It
might seem that we could successfully formulate plans only for systems
we can understand. Instead, every day, programmers successfully
contribute code towards working systems too complex for anyone to
understand \emph{as a whole}. We make use of modularity and
abstraction mechanisms to construct systems whose component plans we
can understand piecemeal, and whose compositions we can understand
without fully understanding each plan being composed.
%
\begin{quote}
Programmers are not to be measured by their ingenuity and their logic
but by the completeness of their case analysis.
\begin{flushright}
---Alan Perlis
\end{flushright}
\end{quote}
%
In the human world, when you plan for yourself, you make assumptions
about future situations in which your plan will unfold. Occasionally,
someone else's plan may interfere with yours, invalidating the
assumptions on which your plan is based. To plan successfully, you
need some sense of which assumptions are usually safe from such
disruption. You do not need to anticipate every possible contingency,
however. If someone does something you did not expect, you will
probably be better able to figure out how to cope at that time anyway.

To formulate plans for machines to execute, programmers must also make
assumptions. When separately formulated plans are composed,
conflicting assumptions can cause the run-time situation to become
\emph{inconsistent} with a given plan's assumptions, leading it
awry. By dividing the state of a computational system into separately
encapsulated objects, and by giving objects limited access to each
other, we limit outside interference and extend the range of
assumptions our programs may safely rely upon.\footnote{
%
This view of encapsulation and composition parallels Hayek's
explanation of how property rights protect human plans from
interference and how trade brings about their cooperative alignment
\cite{Hayek:1945:UKS}. See \cite{miller:agoric,tulloh:abstraction} for
more.}
%
Beyond these assumptions, correct programs must handle all relevant
contingencies. By abstraction, we limit one object's need for
knowledge of others, reducing the number of cases which are
relevant. However, even under sequential and benign conditions, the
remaining case analysis can still be quite painful.

Under concurrency, an object's own plans may destructively interfere
with each other. In distributed programming, asynchrony and partial
failure limit an object's local knowledge of relevant facts,
increasing the number of relevant cases it must consider. In secure
programming, we carefully distinguish those objects whose good
behavior we rely on from those we don't, but we seek to cooperate with
both. Confidentiality further constrains local knowledge; deceit and
malice are further sources of possible plan interference. Each of
these dimensions threatens an explosion of new cases we must
consider. To succeed, we must find ways of reducing the size of the
resulting case analysis.

Previous papers have focused on E's support for limited trust within
the constraints of distributed systems
\cite{miller:ode,miller:myths,miller:paradigm,miller:struct-auth}. This
paper focuses on E's support for concurrent and distributed
programming within the constraints of limited trust.

\section{Overview}

Throughout this paper, we do not seek universal solutions to coordination
problems, but rather, abstraction mechanisms adequate to craft diverse
solutions adapted to the needs of many applications. We illustrate
many of our points with a simple example, a ``\abst{statusHolder}''
object implementing the listener pattern.
%
\begin{description}
\item[The Sequential StatusHolder] introduces the \abst{statusHolder}
and examines its hazards in a sequential environment.

\item[Why Not Shared-state Concurrency] shows several
attempts at a conventionally thread-safe \abst{statusHolder} in
\sys{Java} and the ways each suffers from plan interference.

\item[A Taste of E] shows a \abst{statusHolder} written in E and explains
E's eventual-send operator in the context of a single thread of
control.

\item[Communicating Event-Loops] explains how the \abst{statusHolder}
handles concurrency and distribution under benign conditions.

\item[Protection from Misbehavior] examines how the plans coordinated
by our \abst{statusHolder} are and are not vulnerable to each other.

\item[Promise Pipelining] introduces promises for the results of
eventually-sent messages, and shows how pipelining helps programs
tolerate latency and how broken promise contagion lets programs handle
eventually-thrown exceptions.

\item[Partial Failure] shows how \abst{statusHolder}'s clients can
regain access following a partition or crash and explains the issues
involved in regaining distributed consistency.

\item[The When-Catch Expression] explains how to turn data-flow back
into control-flow.

\item[From Objects to Actors and Back Again] presents a
brief history of E's concurrency control.

\item[Related Work] discusses other systems with similar goals, as
well as current projects adapting these insights to existing
platforms.

\item[Discussion and Conclusions] summarizes current status, what
remains to be done, and lessons learned.

\end{description}

\section{The Sequential \abst{StatusHolder}}

Throughout the paper, we will examine different forms of the listener
pattern \cite{Englander:beans}. The code below is representative
of the basic sequential listener pattern.\footnote{
%
The listener pattern \cite{Englander:beans} is similar to the observer
pattern \cite{gamma:patterns}. However, the analysis which follows
would be quite different if we were starting from the observer
pattern.}
%
In it, a \abst{statusHolder} object is used to coordinate a changing
status between \emph{publishers} and \emph{subscribers}.  A subscriber
can ask for the current status of a \abst{statusHolder} by calling
\meth{getStatus}, or can subscribe to receive notifications when the
status changes by calling \meth{addListener} with a listener object.
A publisher changes the status in a \abst{statusHolder} by calling
\meth{setStatus} with the new value.  This in turn will call
\meth{statusChanged} on all subscribed listeners.  In this way,
publishers can communicate status updates to subscribers without
knowing of each individual subscriber.

We can use this pattern to coordinate several loosely coupled
plans. For example, in a simple application, a bank account manager
publishes an account balance to an analysis spreadsheet and a
financial application.  Deposits and withdrawals cause a new balance
to be published.  The spreadsheet adds a listener that will update the
display to show the current balance. The finance application adds a
listener to begin trading activities when the balance falls below some
threshold.  Although these clients interact cooperatively, they know
very little about each other.
%
\begin{alltt}
    public class \cls{StatusHolder} \{
        private Object \dvar{myStatus};
        private final ArrayList<Listener> \dvar{myListeners} 
                          = new ArrayList();

        public \dmeth{StatusHolder}(Object \dvar{status}) \{
            myStatus = status;
        \}
        public void \dmeth{addListener}(Listener \dvar{newListener}) \{
            myListeners.add(newListener);
        \}
        public Object \dmeth{getStatus}() \{
            return myStatus; 
        \}
        public void \dmeth{setStatus}(Object \dvar{newStatus}) \{
            myStatus = newStatus;
            for (Listener \dvar{listener}: myListeners) \{
                listener.statusChanged(newStatus);
            \}
        \}
    \}
\end{alltt}
%
Even under sequential and benign conditions, this pattern creates plan
interference hazards.
%
\begin{description}

\item[Aborting the wrong plan:] 

If a listener throws an exception, this prevents some other listeners
from being notified of the new status and possibly aborts the
publisher's plan. In the above example, the spreadsheet's inability to
display the new balance should not impact either the finance
application or the bank account manager.

\item[Nested subscription:] 

The actions of a listener could cause a new listener to be subscribed.
For example, to bring a lowered balance back up, the finance
application might initiate a stock trade operation, which adds its own
listener.  Whether that new listener sees the current event, fails to
see the current event, or fails to be subscribed depends on minor
details of the listener implementation.

\item[Nested publication:] 

Similarly, a listener may cause a publisher to publish a new status,
possibly unknowingly due to aliasing.  For example, during an update,
the invocation of \meth{setStatus} notifies the finance application,
which deposits money into the account. A new update to the balance is
published and an inner invocation of \meth{setStatus} notifies all
listeners of the new balance.  After that inner invocation returns,
the outer invocation of \meth{setStatus} continues notifying listeners
of the older, pre-deposit balance.  Some of the listeners would
receive the notifications \emph{out of order}.  As a result, the
spreadsheet might leave the display showing the wrong balance, or
worse, the finance application might initiate transactions based on
incorrect information.

\end{description}
%
The nested publication hazard is especially striking because it
reveals that problems typically associated with concurrency may arise
even in a simple sequential example. This is why we draw attention to
\emph{plans}, rather than programs or processes. The
\abst{statusHolder}, by running each subscriber's plan during a step
of a publisher's plan, has provoked plan interference: these largely
independent plans now interact in surprising ways, creating numerous
new cases that are difficult to identify, prevent, or test. Although
these hazards are real, experience suggests that programmers can
usually find ways to avoid them in sequential programs under benign
conditions.

\section{Why Not Shared-state Concurrency}

With genuine concurrency, interacting plans unfold in parallel. To
manipulate state and preserve consistency, a plan needs to ensure
others are not manipulating that same state at the same time. This
section explores the plan coordination problem in the context of the
conventional shared-state concurrency-control paradigm
\cite{VanRoyHaridi}, also known as shared-memory multi-threading. We
present several attempts at a conventionally \emph{thread-safe}
\abst{statusHolder}---searching for one that prevents its clients from
interfering without preventing them from cooperating.

In the absence of real-time concerns, we can analyze concurrency
without thinking about genuine parallelism. Instead, we can model the
effects of concurrency as the non-deterministic interleaving of atomic
units of operation. We can roughly characterize a
concurrency-control paradigm with the answers to two questions:
%
\begin{description}

\item[Serializability:] 

What are the coarsest-grain units of operation, such that we can
account for all visible effects of concurrency as equivalent to some
fully ordered interleaving of these units \cite{IBM:POO}? For
shared-state concurrency, this unit is generally no larger than a
memory access, instruction, or system call---which is often finer than
the ``primitives'' provided by our programming languages
\cite{boehm:threads}. For databases, this unit is the transaction.

\item[Mutual exclusion:]

What mechanisms can eliminate the possibility of some interleavings,
so as to preclude the hazards associated with them? For shared-state
concurrency, the two dominant answers are monitors
\cite{hoare:monitors,hansen:monitors} and rendezvous
\cite{hoare:csp}. For distributed programming, many systems restrict
the orders in which messages may be delivered
\cite{birman:vsync,amir:thesis,lamport:paxos}.

\end{description}
%
\sys{Java} is loosely in the monitor tradition. \sys{Ada},
\sys{Concurrent ML}, and the synchronous $\pi$-calculus are loosely in
the rendezvous tradition. With minor adjustments, the following
comments apply to both.

\subsection{Preserving Consistency}

If we place our sequential \abst{statusHolder} into a concurrent
environment, publishers or subscribers may call it from different
threads. The resulting interleaving of operations might, for example,
mutate the \var{myListeners} list while the for-loop is in progress.

\begin{figure}
\centerline{\epsfig{figure=seesaw.eps}}
\caption{A correct program must both remain consistent and continue to
make progress. The sequence above represents our search for a
\abst{statusHolder} which supports both well:
(1) The sequential \abst{statusHolder}. 
(2) The sequential \abst{statusHolder} in a concurrent environment.
(3) The fully synchronized \abst{statusHolder}.
(4) Placing the for-loop outside the synchronized block.
(5) Spawning a new thread per listener notification.
(6) Using communicating event-loops}
\label{fig:seesaw}
\end{figure}

Adding the ``\code{synchronized}'' keyword to all methods of the above
code causes it to resemble a monitor. This fully synchronized
\abst{statusHolder} eliminates exactly those cases where multiple
plans interleave within the \abst{statusHolder}. It is as good at
preserving its own consistency as our original sequential
\abst{statusHolder} was.

However, it is generally recommended that \sys{Java} programmers avoid
this fully synchronized pattern because it is prone to deadlock
\cite{Englander:beans}. Although each listener is called from some
publisher's thread, its purpose may be to contribute to a plan
unfolding in its subscriber's thread. To defend itself against such
concurrent entry, the objects at this boundary may themselves be
synchronized. If a \meth{statusChanged} notification gets blocked
here, waiting on that subscriber's thread, it blocks the
\abst{statusHolder}, as well as any other objects whose locks are held
by that publisher's thread. If the subscriber's thread is itself
waiting on one of these objects, we have a classic deadly embrace.

Although we have eliminated interleavings that lead to inconsistency,
some of the interleavings we eliminated were necessary to make
progress.

\subsection{Avoiding Deadlock}

To avoid this problem, \cite{Englander:beans} recommends changing the
\meth{setStatus} method to clone the listeners list within the
synchronized block, and then to exit the block before entering the
for-loop, as shown by the code below. This pattern avoids holding a
lock during notification and thus avoids the obvious deadlock
described above between a publisher and a subscriber.  It does not
avoid the underlying hazard, however, because the publisher may hold
other locks.
%
\begin{alltt}
    public void \dmeth{setStatus}(Object \dvar{newStatus}) \{
        ArrayList<Listener> \dvar{listeners};
        synchronized (this) \{
            myStatus = newStatus;
            listeners = (ArrayList<Listener>)myListeners.clone();
        \}
        for (Listener \dvar{listener}: listeners) \{
            listener.statusChanged(newStatus);
        \}
    \}
\end{alltt}
%
For example, if the account manager holds a lock on the bank account
during a withdrawal, a deposit attempt by the finance application
thread may result in an equivalent deadlock, with the account manager
waiting for the notification of the finance application to complete,
and the finance application waiting for the account to unlock.  The
result is that all the associated objects are locked and other
subscribers will never hear about this update.  Thus, the underlying
hazard remains.

In this approach, some interleavings needed for progress are still
eliminated, and as we will see, some newly-allowed interleavings lead
to inconsistency.

\subsection{Race Conditions}

The approach above has a consistency hazard: if \meth{setStatus} is
called from two threads, the order in which they update \var{myStatus}
will be the order they enter the synchronized block above. However,
the for-loop notifying listeners of a later status may race ahead of
one that will notify them of an earlier status. As a result, even a
single subscriber may see updates out of order, so the spreadsheet may
leave the display showing the wrong balance, even in the absence of
any nested publication.

It is possible to adjust for these remaining problems. The style
recommended for some rendezvous-based languages, like \sys{Concurrent
ML} and the $\pi$-calculus, corresponds to spawning a separate thread
to perform each notification.  This avoids using the producer's thread
to notify the subscribers and thus avoids the deadlock
hazard---it allows all interleavings needed for progress. However, this
style still suffers from the same race condition hazards and so still
fails to eliminate the right interleavings. We could compensate for this
by adding a counter to the \abst{statusHolder} and to the notification
API, and by modifying the logic of all listeners to reorder
notifications. But a formerly trivial pattern has now exploded into a
case-analysis minefield. Actual systems contain thousands of patterns
more complex than the \abst{statusHolder}. Some of these will suffer
from less obvious minefields.
%
\begin{quotation}
This is ``Multi-Threaded Hell''. As your application evolves, or as
different programmers encounter the sporadic and non-reproducible
corruption or deadlock bugs, they will add or remove locks around
different data structures, causing your code base to veer back and
forth \ldots, erring first on the side of more deadlocking, and then
on the side of more corruption. This kind of thrashing is bad for the
quality of the code, bad for the forward progress of the project, and
bad for morale.
\begin{flushright}
---An experience report from the development of Mojo Nation \cite{zooko:hell}
\end{flushright}
\end{quotation}

\section{A Taste of E}

Before revisiting the issues above, let's first use this example to
briefly explain E as a sequential object language. (For a more
complete explanation of E, see \cite{stiegler:ewalnut}.)  Here is the
same \abst{statusHolder} as defined in E.
%
\begin{alltt}
    def \dobj{makeStatusHolder}(var \dvar{myStatus}) \{
        def \dvar{myListeners} := [].diverge()
        def \dobj{statusHolder} \{
            to \dmeth{addListener}(\dvar{newListener}) \{
                myListeners.push(newListener)
            \}
            to \dmeth{getStatus}() \{ return myStatus \}
            to \dmeth{setStatus}(\dvar{newStatus}) \{
                myStatus := newStatus
                for \dvar{listener} in myListeners \{
                    listener.statusChanged(newStatus)
                \}
            \}
        \}
        return statusHolder
    \}
\end{alltt}
%
E has no classes. Instead, the expression beginning with ``\code{def
\dobj{statusHolder}}'' is an object definition expression. It creates
a new object with the enclosed method definitions and binds the new
\var{statusHolder} variable to this object. An invocation, such as
``\code{statusHolder.setStatus(33)}'', causes a message to be
delivered to an object. When an object receives a message, it reacts
according to the code of its matching method. As with \sys{Smalltalk}
\cite{goldberg:purplebook} or \sys{Actors} \cite{hewitt:actors}, all
values are objects, and all computation proceeds only by delivering
messages to objects.

% Keep the space before ``From'' below to protect from email.
 From a $\lambda$-calculus perspective, an object definition
expression is a lambda expression, in which the (implicit) parameter
is bound to the incoming message and the body selects a method to run
according to the message. The delivery of a message to an object is
the application of an object-as-closure to a message-as-argument. An
object's behavior is indeed a function of the message it is applied
to. This view of objects goes back to \sys{Smalltalk-72}
\cite{goldberg:smalltalk72} and \sys{Actors}, and is hinted at earlier
in \cite{hoare65}. Also see \cite{shroff:match}.

Unlike a class definition, an object definition does not declare its
instance variables. Instead, the instance variables of an object are
simply the variables used freely within the object definition (which
therefore must be defined in some lexically enclosing scope). The
instance variables of \abst{statusHolder} are \var{myStatus} and
\var{myListeners}.  Variables are unassignable by default; the
``\code{var}'' keyword defines \var{myStatus} as an assignable
variable. Square brackets evaluate to an immutable list containing the
values of the subexpressions (the empty-list in the example). Lists
respond to the ``\code{diverge()}'' message by returning a new mutable
list whose initial contents are a snapshot of the diverged list. Thus,
\var{myListeners} is initialized to a new, empty, mutable list, which
acts much like an \cls{ArrayList}.

\sys{E} provides syntactic shorthands to use objects that define a
``\meth{run}'' method as if they were functions. The syntax for
\var{makeStatusHolder} is a shorthand for defining an object with a
single ``\meth{run}'' method. It expands to:
%
\begin{alltt}
    def \dobj{makeStatusHolder} \{
        to \dmeth{run}(var \dvar{myStatus}) \{ ...
\end{alltt}
%
The corresponding function call syntax,
``\code{makeStatusHolder(44)}'', is shorthand which expands to
``\code{makeStatusHolder.run(44)}''. Each time \var{makeStatusHolder} is
called, it defines and returns a new \abst{statusHolder}.

\subsection{Two Ways to Postpone Plans}

The E code for \abst{statusHolder} above retains the simplicity and
hazards of the sequential Java version.  To address these hazards
requires examining the underlying issues.  When the
\abst{statusHolder}---or any agent---is executing plan $X$ and
discovers the need to engage in plan $Y$, in a sequential system, it
has two simple alternatives of when to do $Y$:
%
\begin{description}

\item[Immediately:] Put $X$ aside, work on $Y$ until
complete, then go back to $X$.

\item[Eventually:] Put $Y$ on a ``to-do'' list and work on it after
$X$ is complete.

\end{description}
%
The ``immediate'' option corresponds to conventional, sequential
call-return control flow (or strict applicative-order evaluation), and
is represented by the ``\code{.}'' or \emph{immediate-call} operator,
which delivers the message immediately. Above, \abst{statusHolder}'s
\meth{addListener} method tells \var{myListeners} to push the
\var{newListener} \emph{immediately}. When \meth{addListener} proceeds past
this point, it may assume that all side effects it requested are done.

For the \abst{statusHolder} example, all of the sequential hazards
(e.g., Nested Publication) and many of the concurrent hazards
(deadlock) occur because the \meth{statusChanged} method is also
invoked immediately: the publisher's plan is set aside to pursue the
listener's plan (which might then abort, change the state further,
etc.).

The ``eventual'' option corresponds to the human notion of a ``to-do''
list: the item is queued for later execution. E provides direct
support for this asynchronous messaging option, represented by the
``\code{<-}'' or \emph{eventual-send} operator. Using eventual-send,
the \meth{setStatus} method can ensure that each listener will be
notified of the changed status in such a way that it does not
interfere with the \abst{statusHolder}'s current plan.  To accomplish
this in E, the \meth{setStatus} method becomes:
%
\begin{alltt}
    to \dmeth{setStatus}(\dvar{newStatus}) \{
        myStatus := newStatus
        for \dvar{listener} in myListeners \{
            listener <- statusChanged(newStatus)
        \}
    \}
\end{alltt}
%
As a result of using eventual-send above, all of the sequential
hazards are addressed. Errors, new subscriptions, and additional
status changes caused by listeners will all take place after all
notifications for a published event have been scheduled.  Publishers'
plans and subscribers' plans are temporally isolated---so these
plans may unfold with fewer unintended interactions. For example, it
can no longer matter whether \var{myStatus} is assigned before or
after the for-loop.

\subsection{Simple E Execution}

This section describes how temporal isolation is achieved within a
single thread of control.  The next section describes how it is
achieved in the face of concurrency and distribution.

\begin{figure}
\centerline{\epsfig{figure=big1vat.eps}}
\caption{An E vat consists of a heap of objects and a thread of
  control. The stack and queue together record the postponed plans the
  thread needs to process. An immediate-call pushes a new frame on top
  of the stack, representing the delivery of a message ({\it arrow})
  to a target object ({\it dot}). An eventual-send enqueues a new
  pending delivery on the right end of the queue. The thread proceeds
  from top to bottom and then from left to right}
\label{fig:stackvat}
\end{figure}

In E, an eventual-send creates and queues a \emph{pending delivery},
which represents the eventual delivery of a particular message to a
particular object.  Within a single thread of control, E has both a
normal execution stack for immediate call-return and a queue
containing all the pending deliveries. Execution proceeds by taking a
pending-delivery from the queue, delivering its message to its object,
and processing all the resulting immediate-calls in conventional
call-return order.  This is called a \emph{turn}.  When a pending
delivery completes, the next one is dequeued, and so forth.  This is
the classic event-loop model, in which all of the events are pending
deliveries. Because each event's turn runs to completion before the
next is serviced, they are temporally isolated.

Additional mechanisms to process results and exceptions from
eventual-sends will be discussed in further sections below.

The combination of a stack, a pending delivery queue, and the heap of
objects they operate on is called a \emph{vat}, illustrated in
Figure~\ref{fig:stackvat}.\footnote{
%
Figures~\ref{fig:stackvat}--\ref{fig:refstates} were created by
Ka-Ping Yee with input from the e-lang community.}
%
Each E object lives in exactly one vat and a vat may host many
objects.  Each vat lives on one machine at a time and a machine may
host many vats. The vat is also the minimum unit of persistence,
migration, partial failure, resource control, and defense from denial
of service. We will return to some of these topics below.

\section{Communicating Event-Loops}

We now consider the case where our account (including account manager
and its \abst{statusHolder}) runs in \vat{A} on one machine, and our
spreadsheet (including its listener) runs in \vat{S} on another
machine.

In E, we distinguish several reference-states. A direct reference
between two objects in the same vat is a \emph{near
reference}.\footnote{
%
For brevity, we generally do not distinguish a near reference from the
object it designates.}
%
As we have seen, near references carry both immediate-calls and
eventual-sends. Only \emph{eventual references} may cross vat
boundaries, so the spreadsheet holds an eventual reference to the
\abst{statusHolder}, which in turns holds an eventual reference to the
spreadsheet's listener. Eventual references are first class---they can
be passed as arguments, returned as results, and stored in data
structures, just like near references. However, eventual references
carry only eventual-sends, not immediate-calls---an immediate-call on
an eventual reference throws an exception. Our \abst{statusHolder} is
compatible with this constraint, since it stores, retrieves, and
eventual-sends to its listeners, but never immediate-calls them.
Figure~\ref{fig:2vat} shows what happens when a message is sent
between vats.

\begin{figure}
\centerline{\epsfig{figure=big2vat.eps,width=340pt}}
\caption{If the account manager and the spreadsheet are in separate
  vats, when the account manager (1) tells the \abst{statusHolder}
  that represents its balance to immediately update, this (2)
  transfers control to the \abst{statusHolder}, which (3) notes that
  its listeners should eventually be notified. The message is (4) sent
  to the spreadsheet's vat, which queues it on arrival and eventually
  (5) delivers it to the listener, which updates the display of the
  spreadsheet cell}
\label{fig:2vat}
\end{figure}

When the \abst{statusHolder} in \vat{A} performs an eventual-send of
the \meth{statusChanged} message to the spreadsheet's listener in
\vat{S}, \vat{A} creates a pending delivery as before, recording the
need to deliver this message to this listener. Pending deliveries need
to be queued on the pending delivery queue of the vat hosting the
object that will receive the message---in this case, \vat{S}. \vat{A}
serializes (marshals) the pending delivery onto an encrypted,
order-preserving byte stream read by \vat{S}. Should it ever arrive at
\vat{S}, \vat{S} will unserialize it and queue it on its own pending
delivery queue.

Since each vat runs concurrently with all other vats, turns in
different vats no longer have actual temporal isolation. If \vat{S} is
otherwise idle, it may service this delivery, notifying the
spreadhseet's listener of the new balance, while the original turn is
still in progress in \vat{A}. But so what? These two turns can only
execute simultaneously when they are in different vats. In this case,
the spreadsheet cannot affect the account manager's
turn-in-progress. Because only eventual references span between vats,
the spreadsheet can only affect \vat{A} by eventual-sending to objects
hosted by \vat{A}. This cannot affect any turn already in progress in
\vat{A}---\vat{A} only queues the pending delivery, and will service
it sometime after the current turn and turns for previously queued
pending deliveries, complete.

Only near references provide one object synchronous access to
another. Therefore an object has synchronous access to state only
within its own vat. Taken together, these rules guarantee that a
running turn---a sequential call-return program---has mutually
exclusive access to everything to which it has synchronous access. In
the absence of real-time concerns, this provides all the isolation
that was achieved by temporal isolation in the single-threaded case.

The net effect is that a turn is E's unit of operation. We can
faithfully account for the visible effects of concurrency without any
interleaving of the steps within a turn. Any actual multi-vat
computation is equivalent to some fully ordered interleaving of
turns.\footnote{
%
An E turn may never terminate, which is hard to account for within
this simple model of serializability. There are formal models of
asynchronous systems that can account for non-terminating events
\cite{chandy:snapshots}. Within the scope of this paper, we can safely
ignore this issue.

The actual E system does provide synchronous file I/O operations. When
these files are local, prompt, and private to the vat accessing them,
this does not violate turn isolation, but since files may be remote,
non-prompt, or shared, the availability of these synchronous I/O
operations does violate the E model.}
%
Because E has no explicit locking constructs, computation within a
turn can never block---it can only run, to completion or forever. A
vat as a whole is either processing pending deliveries, or is idle
when there are no pending deliveries to service. Because computation
never blocks, it cannot deadlock.  Other lost progress hazards are
discussed in the section on ``Datalock'' below.

As with database transactions, the length of an E turn is not
predetermined. It is a tradeoff left for the developer to decide. How
the object graph is carved up into vats and how computation is carved
up into turns will determine which interleaving cases are eliminated,
and which must be handled explicitly by the programmer. For example,
when the spreadsheet was co-located with the \abst{statusHolder}, it
could immediate-call both \meth{getStatus} and \meth{addListener} in
order to ensure that the spreadsheet's cell sees exactly the updates
to an initial valid state. But when it can only eventual-send these
messages, they may arrive at the \abst{statusHolder} interleaved with
other messages. To relieve potentially remote clients of this burden,
the \abst{statusHolder} should send an initial notification to newly
subscribed listeners:
%
\begin{alltt}
    to \dmeth{addListener}(\dvar{newListener}) \{
        myListeners.push(newListener)
        newListener <- statusChanged(myStatus)
    \}
\end{alltt}
%

\subsection{Issues with Event-loops}

This architecture imposes some strong constraints on programming
(e.g., no threads or coroutines), which can impede certain useful
patterns of plan cooperation. In particular, recursive algorithms,
such as recursive-descent parsers, must a) happen entirely within a
single turn, b) be redesigned (e.g., as a table-driven parser), or c)
if it needs external non-prompt input (e.g., a stream from the user),
be run in a dedicated vat. E programs have used each of these
approaches.

Thread-based coordination patterns can typically be adapted to vat
granularity.  For example, rather than adding the complexity of a
priority queue for pending deliveries, different vats would simply run
at different processor priorities. For example, if a user-interaction
vat \emph{could} proceed (has pending deliveries in its queue), it
should; a helper ``background'' vat (e.g., spelling check) should
consume processor resources only if no user-directed action could
proceed. A divide-and-conquer approach for multi-processing could run a
vat on each processor and divide the problem among them. The
event-loop approach is unsuitable for problems that cannot easily be
adapted to a message-passing hardware architecture, such as fluid
dynamics computation.

\section{Protection from Misbehavior}

When using a language that supports shared-state concurrency, one can
choose to avoid it and adopt the event-loop style instead. Indeed,
several \sys{Java} libraries, such as \sys{AWT}, were initially
designed to be thread-safe, and were then redesigned around
event-loops. Using event-loops, one can easily write a \sys{Java}
class equivalent to our \var{makeStatusHolder}.  If one can so easily
choose to avoid shared-state concurrency, does E actually need to
prohibit it?

E uses the event-loop approach to simplify the task of preserving
consistency while maintaining progress. Preserving consistency stays
simple for the \abst{statusHolder} only if it executes in at most one
thread at a time.  As we discussed previously, the possibility of
multiple threads would necessitate complex locking. If one of its clients
\emph{could} create a new thread and call it, then the simple version
of the \abst{statusHolder} could not preserve consistency (i.e., it
would need to perform the complex locking mentioned in the previous
section). 

In the extreme case, one object may actively intend to disrupt the plans
of another.  This leads us to examine plan coordination in the
presence of malicious behavior.  The topic is of interest both because
large and distributed systems in practice need to handle potentially
malicious components, and because analysis of the malicious case can
help uncover hazards that are already present in the non-malicious case.

\subsection{Defensive Correctness}

If a user browsing a webserver were able to cause incorrect pages to
be displayed to other users, we would likely consider it a bug in the
webserver---we expect it to remain correct regardless of the client's
behavior.  We call this property \emph{defensive correctness}: a
program \name{P} is defensively correct if it remains correct despite
arbitrary behavior on the part of its clients.  Before this definition
can be useful, we need to pin down what we mean by ``arbitrary''
behavior.

When we say that a program \name{P} is correct, this normally means
that we have a specification in mind, and that \name{P} behaves
according to that specification.  There are some implicit caveats in
that assertion. For example, \name{P} cannot behave at all unless it
is run on a machine; if the machine operates incorrectly, \name{P} on
that machine may behave in ways that deviate from its specification. 
We do not consider this to be a bug in \name{P}, because \name{P}'s
correctness implicitly depends on the machine's correctness.  If
\name{P}'s correctness depends on another component \name{R}'s
correctness, we will say that \name{P} \emph{relies upon} \name{R}.
For example, a typical webserver relies on the underlying machine and
on operating system features such as files and sockets.  We will refer
to the set of all elements on which \name{P} relies as \name{P}'s
\emph{reliance set}.\footnote{
%
     The set of all things that \name{P} relies on is similar in
     concept to \name{P}'s ``Trusted Computing Base'' or TCB. ``Rely''
     articulates the objective situation (\name{P} is vulnerable to
     \name{R}), and so avoids confusions engendered by the word
     ``trust''.
     
     While the focus in this paper is on correctness, a similar ``reliance''
     analysis could be applied to other program properties, such as
     promptness \cite{hardy:keykos}.}

We define \name{Q}'s \emph{authority} as the set of effects \name{Q}
could cause.  With regard to \name{P}'s correctness, \name{Q}'s
\emph{relevant authority} is bounded by the assumption that everything
in \name{P}'s reliance set is correct, since \name{P} was defined
under this assumption. For example, if a user could cause a webserver
to show the wrong page to other browsers by replacing a file through
an operating system exploit, then the underlying operating system
would be incorrect, not the webserver. We say that \name{P}
\emph{protects against} \name{Q} if \name{P} remains correct despite
any of the effects in \name{Q}'s relevant authority, that is, despite
any possible actions by \name{Q}, assuming the correctness of
\name{P}'s reliance set.

Now we can speak more precisely about defensive correctness. The
``arbitrary behavior'' mentioned earlier is the combined relevant
authority of an object's clients. \name{P} is \emph{defensively
correct} if it protects against all of its clients.  The focus is on
\emph{clients} in particular in order to enable the composition of
correct components into larger correct systems. If \name{P} relies on
\name{R}, then \name{P} also relies on all of \name{R}'s other clients
\emph{unless} \name{R} is defensively correct.  If \name{R} does not
protect against its other clients, \name{P} cannot prevent them from
interfering with its own plan, which makes it infeasible for \name{P}
to ensure its own correctness.  By not relying on its clients,
\name{R} enables them to avoid relying on each other.  

This explains why it is important for E to forbid the spawning of
threads.  As we saw earlier, it can be very difficult to write
programs in which threads protect against each other.  Removing
threads eliminates a key obstacle to defensive correctness.

Correctness can be divided into consistency (safety) and progress
(liveness). An object that is vulnerable to denial-of-service by its
clients may nevertheless be \emph{defensively consistent}. Given that
all the objects it relies on themselves remain consistent, a
defensively consistent object will never give incorrect service to
well-behaved clients, but it may be prevented from giving them any
service. While a defensively correct object is invulnerable to its
clients, a defensively consistent object is merely incorruptible by
its clients.

Different security properties are feasible at different
granularities. Some conventional operating systems attempt to provide
support for protecting users from each other's misbehavior. But
because programs are normally run with their user's full authority,
all software run under the same account is mutually reliant: since
each is granted the authority to corrupt the others via underlying
components on which they all rely, they cannot usefully protect
against such ``friendly fire''.\footnote{
%
See \cite{stiegler:polaris} for an unconventional way to use
conventional OSes to provide greater security.}
%
Some operating system designs \cite{dvh} support process-granularity
defensive consistency. Others, by providing principled controls over
computational resource rights \cite{hardy:keykos,shapiro:eros}, can
also protect against denial of service. Among machines distributed
over today's Internet, cryptographic protocols help support defensive
consistency, but defensive correctness remains infeasible.

In most programming languages, all objects in the same process are
mutually reliant. A secure language is one which supports some useful
form of protections within a process.  Among objects in the same vat,
E supports defensive consistency: Any object may go into an infinite
loop, thereby preventing the progress of all other objects within
their vat. Therefore, within E's architecture, defensive correctness
\emph{within} a vat is impossible. With respect to progress, all
objects within the same vat are mutually reliant. In many situations,
defensive consistency is adequate---a potential adversary often has
more to gain from corruption than denial of service. This is
especially so in iterated relationships, since corruption may
misdirect plans but go undetected, while loss of progress is quite
noticeable.

\subsection{Principle of Least Authority (POLA)}

Our \abst{statusHolder} itself is now defensively consistent, but is
it a good abstraction for the account manager to rely on to build its
own defensively consistent plans? In our example scenario, we have
been assuming that the account manager acts only as a publisher and
that the finance application and spreadsheet act only as
subscribers. However either subscriber \emph{could} invoke the
\meth{setStatus} method. If the finance application calls
\meth{setStatus} with a bogus balance, the spreadsheet will dutifully
render it.

This is a problem of access control. The \abst{statusHolder}, by
bundling two kinds of authority into one object, encouraged patterns
where both kinds of authority were provided to objects that only
needed one. This can be addressed by grouping these methods into
separate objects, each of which represents a sensible bundle of
authority.
%
\begin{alltt}
    def \dobj{makeStatusPair}(var \dvar{myStatus}) \{
        def \dvar{myListeners} := [].diverge()
        def \dobj{statusGetter} \{
            to \dmeth{addListener}(\dvar{newListener}) \{
                myListeners.push(newListener)
                newListener <- statusChanged(myStatus)
            \}
            to \dmeth{getStatus}() \{ return myStatus \}
        \}
        def \dobj{statusSetter} \{
            to \dmeth{setStatus}(\dvar{newStatus}) \{
                myStatus := newStatus
                for \dvar{listener} in myListeners \{
                    listener <- statusChanged(newStatus)
                \}
            \}
        \}
        return [statusGetter, statusSetter]
    \}
\end{alltt}
%
Now the account manager can make use of \var{makeStatusPair} as
follows:
%
\begin{alltt}
    def [\dvar{sGetter}, \dvar{sSetter}] := makeStatusPair(33)
\end{alltt}
%
The call to \var{makeStatusPair} on the right side makes four
objects---an object representing the \var{myStatus} variable, a
mutable \var{myListeners} list, a \var{statusGetter}, and a
\var{statusSetter}. The last two each share access to the first
two. The call to \var{makeStatusPair} returns a list holding these
last two objects. The left side pattern-matches this list, binding
\var{sGetter} to the new \var{statusGetter}, and binding \var{sSetter}
to the new \var{statusSetter}.

The account manager can now keep the new \var{statusSetter} for
itself and give the spreadsheet and the finance application access
only to the new \var{statusGetter}. More generally, we may now
describe publishers as those with access to \var{statusSetter} and
subscribers as those with access to \var{statusGetter}. The account
manager can now provide consistent balance reports to its clients
because it has denied them the possibility of corrupting this
service. 

As with concurrency control, the key to access control is to allow the
possibilities needed for cooperation, while limiting the possibilities
that would allow for plan interference. We wish to provide
objects the authority needed to carry out their proper
duties---publishers gotta publish---but little more. This is known as
\emph{POLA}, the \emph{Principle of Least Authority} (See
\cite{miller:paradigm} for the relationship between POLA and the
Principle of Least Privilege \cite{SaltzerSc75}). By not granting its
subscribers the authority to publish a bogus balance, the account
manager no longer needs to worry about what would happen if they
did. This discipline helps us compose plans so as to allow
well-intentioned plans to successfully cooperate, while minimizing the
kinds of plan interference they must defend against.

\subsection{A Taste of E across a Network}

E's computational model extends across the network.
An eventual reference in a vat can refer to an object in
a vat on another machine; eventual-sends to that reference are sent
across an encrypted, authenticated link and posted as pending
deliveries for the target object on the remote vat.

E's network protocol, \sys{Pluribus}, actually runs between vats, not
between machines. Therefore, we can ignore the distinction between
vats and machines without loss of generality. An incorrect machine
is, from our perspective, simply a set of incorrect vats; i.e., vats
that do not implement the language and/or protocol correctly. The
design of \sys{Pluribus} is beyond the scope of this document, but a
few words are in order.

\sys{Pluribus} enforces characteristics of the E computational model,
such as reference integrity, so that E programs can rely on those
properties between vats and therefore between machines. Even if a
remote vat runs its objects in an unsafe language like C++, other vats
could still view it from a correctness point of view as a set of
(possibly incorrect) objects written in E. From the perspective of
other vats, the objects in the remote vat could collude and act
arbitrarily within the union of the authorities granted to any of
them, but they cannot feasibly\footnote{
%
\sys{Pluribus} relies on the standard cryptographic assumptions that
large random numbers are not feasibly guessable, and that
well-accepted algorithms are immune to feasible cryptanalysis.}
%
manufacture new authorities. Thus, if an object relies on another object in
a remote vat, then it also relies on that remote vat (because the
remote object relies on that vat).

\section{Promise Pipelining}

The eventual-send examples so far were carefully selected to be
evaluated only for their effects, with no use made of the value of
these expressions. This section discusses the handling of return
results and exceptions produced by eventual-sends.

\subsection{Promises}

As discussed previously, eventual-sends queue a pending delivery and
complete immediately.  The return value from an eventual-send
operation is called a \emph{promise} for the eventual result. The
promise is not a near reference for the result of the eventual-send
because the eventual-send cannot have happened yet (i.e., it will
happen in a later turn). Instead, the promise is an eventual-reference
for the result.  A pending delivery, in addition to the message and
reference to the target object, includes a \emph{resolver} for the
promise, which provides the right to choose what the promise
designates. When the turn spawned by the eventual-send completes, its
vat reports the outcome to the resolver, \emph{resolving} the promise
so that the promise eventually becomes a reference designating that
outcome, called the \emph{resolution}.  

Once resolved, the promise is equivalent to its resolution. Thus, if
it resolves to an eventual-reference for an object in another vat,
then the promise becomes that eventual reference. If it resolves to
an object that can be passed by copy between vats, then it becomes a
near-reference to that object.

Because the promise starts out as an eventual reference, messages can
be eventually-sent to it even \emph{before} it is resolved. Messages
sent to the promise cannot be delivered until the promise is resolved,
so they are buffered in FIFO order within the promise. Once the
promise is resolved, these messages are forwarded, in order, to its
resolution.

\subsection{Pipelining}

Since an object can eventual-send to the promises resulting from
previous eventual-sends, functional composition is
straightforward. If object \name{L} in \vat{L} executes
%
\begin{alltt}
    def \dvar{r3} := x <- a() <- c(y <- b())
\end{alltt}
%
or equivalently
%
\begin{alltt}
    def \dvar{r1} := x <- a()
    def \dvar{r2} := y <- b()
    def \dvar{r3} := r1 <- c(r2)
\end{alltt}
%
and \var{x} and \var{y} are on \vat{R}, then all three requests are
serialized and streamed out to \vat{R} immediately and the turn in
\vat{L} continues without blocking. By contrast, in a conventional RPC
system, the calling thread would only proceed after multiple network
round trips.

\begin{figure}
\centerline{\epsfig{figure=pipeline-only.eps}}
\caption{The three messages in \code{def \dvar{r3} := x <- a()
<- c(y <- b())} are streamed out together, with no round
trip. Each message box ``rides'' on the reference it is sent
on. References \var{x} and \var{y} are shown with solid arrowheads,
indicating that their target is known. The others are \emph{promises},
whose open arrowhead represents their \emph{resolvers}, which provide
the right to choose their promises' value}
\label{fig:pipeline}
\end{figure}

Figure~\ref{fig:pipeline} depicts an unresolved reference as an arrow
stretching between its promise-end, the tail held by \var{r1}, and its
resolver, the open arrowhead within the pending delivery sent to
\vat{R}. Messages sent on a reference always flow towards its
destination and so ``move'' as close to the arrowhead as
possible. While the pending delivery for \code{a()} is in transit to
\vat{R}, so is the resolver for \var{r1}, so we send the \code{c(r2)}
message there as well. As \vat{R} unserializes these three requests,
it queues the first two in its local to-do list, since their target is
known and local. It sends the third, \code{c(r2)}, on a local
promise that will be resolved by the outcome of \code{a()}, carrying
as an argument a local promise for the outcome of \code{b()}.

If the resolution of \var{r1} is local to \vat{R}, then as soon as
\code{a()} is done, \code{c(r2)} is immediately queued on \vat{R}'s
to-do list and may well be serviced before \vat{L} learns of
\var{r1}'s resolution. If \var{r1} is on \vat{L}, then \code{c(r2)} is
streamed back towards \vat{L} just behind the message informing
\vat{L} of \var{r1}'s resolution.  If \var{r1} is on yet a third vat,
then \code{c(r2)} is forwarded to that vat.

Across geographic distances, latency is already the dominant
performance consideration. As hardware improves, processing will
become faster and cheaper, buffers larger, and bandwidth greater, with
limits still many orders of magnitude away. But latency will remain
limited by the speed of light. Pipes between fixed endpoints can be
made wider but not shorter. Promise pipelining reduces the impact of
latency on remote communication. Performance analysis of this type of
protocol can be found in Bogle's ``Batched Futures''
\cite{bogle:batched}; the promise pipelining protocol is approximately
a symmetric generalization of it.

\subsection{Datalock}

Promise chaining allows some plans, like \code{c(r2)}, to be postponed
pending the resolution of previous plans. We introduce other ways to
postpone plans below.  Using the primitives introduced so far,
however, it is possible to create circular data dependencies which,
like deadlock, are a form of lost-progress bug. We call this kind of
bug, \emph{datalock}.  For example, the \var{epimenides} function below
returns a promise for the boolean opposite of \var{flag}.
%
\begin{alltt}
    var \dvar{flag} := true
    def \dobj{epimenides}() \{ return flag <- not() \}
\end{alltt}
%
If \var{flag} were assigned to the result of invoking
\var{epimenides} eventually, datalock would occur.
%
\begin{alltt}
    flag := epimenides <- run()
\end{alltt}
%
In the current turn, a pending-delivery of \code{epimenides <- run()}
is queued, and a promise for its result is immediately assigned to
\var{flag}.  In a later turn when \var{epimenides} is invoked, it
eventual-sends a message to the promise in \var{flag}, and then
resolves the \var{flag} promise to the new promise for the
\code{not()} sent to that \emph{same} \var{flag} promise.  The datalock is
created, not because a promise is resolved to another promise (which
is acceptable and common), but because computing the eventual
resolution of \var{flag} requires already knowing it.

Although the E model trades one form of lost-progress bug for another,
it is still more reliable. As above, datalock bugs primarily represent
circular dependencies in the computation, which manifest reproducibly
like normal program bugs. This avoids the significant non-determinism,
non-reproducibility, and resulting debugging difficulty of deadlock
bugs. Anecdotally, in many years of programming in E and E-like
languages and a body of experience spread over perhaps 60 programmers
and two substantial distributed systems, we know of only two datalock
bugs. Perhaps others went undetected, but these projects did not spend
the agonizing time chasing deadlock bugs that projects of their nature
normally must spend.  Further analysis is needed to understand why datalock
bugs seem to be so rare.

\subsection{Explicit Promises}

Besides the implicit creation of promise-resolver pairs by
eventual-sending, E provides a primitive to create these pairs
explicitly. In the following code
%
\begin{alltt}
    def [\dvar{p}, \dvar{r}] := Ref.promise()
\end{alltt}
%
\var{p} and \var{r} are bound to the promise and resolver of a new
promise/resolver pair. Explicit promise creation gives us yet greater
flexibility to postpone plans until other conditions occur. The
promise, \var{p}, can be handed out and used just as any other
eventual reference. All messages eventually-sent to \var{p} are queued
in the promise. An object with access to \var{r} can wait until some
condition occurs before resolving \var{p} and allowing these pending
messages to proceed, as a later example will demonstrate.

\subsection{Broken Promise Contagion}

Because eventual-sends are executed in a later turn, an exception
raised by one can no longer signal an exception and abort the plan of
its ``caller''. Instead, the vat executing the turn for the eventual
send catches any exception that terminates that turn and
\emph{breaks} the promise by resolving the promise to a \emph{broken
reference} containing that exception.  Any immediate-call or
eventual-send to a broken reference breaks the result with the broken
reference's exception.  Specifically, an immediate-call to a broken
reference would throw the exception, terminating control flow.  An
eventual-send to a broken reference would break the eventual-send's
promise with the broken reference's exception. As with the original
exception, this would not terminate control flow, but does affect
plans dependent on the resulting value.

E's split between control-flow exceptions and data-flow exceptions was
inspired by signaling and non-signaling NaNs in floating point. Like
non-signaling NaNs, broken promise contagion does not hinder
pipelining. Following sections discuss how additional sources of
failure in distributed systems cause broken references, and how E
handles them while preserving defensive consistency.

\section{Partial Failure}

Not all exceptional conditions are caused by program behavior.
Networks suffer outages, partitioning one part of the network from
another. Machines fail: sometimes in a transient fashion, rolling back
to a previous stable state; sometimes permanently, making the objects
they host forever inaccessible. From a machine not able to reach a
remote object, it is generally impossible to tell which failure is
occurring or which messages were lost.

Distributed programs need to be able to react to these conditions so
that surviving components can continue to provide valuable and
correct---though possibly degraded---service while other components are
inaccessible. If these components may change state while out of
contact, they must recover distributed consistency when they
reconnect. There is no single best strategy for maintaining
consistency in the face of partitions and merges; the 
appropriate strategy will depend on the semantics of the components. A general
purpose framework should provide simple mechanisms adequate to express
a great variety of strategies. Group membership and similar systems
provide one form of such a general framework, with strengths and
weaknesses in comparison with E. Here, we explain E's framework. We
provide a brief comparison with mechanisms like group membership in
the ``Related Work'' section below.

E's support for partial failure starts by extending the semantics of
our reference states. Figure~\ref{fig:refstates} shows the full state
transition diagram among these states.

\begin{figure}
\centerline{\epsfig{figure=refstates2.eps}}
\caption{A resolved reference's target is known. Near
  references are resolved and local; they carry both immediate-calls
  and eventual-sends. Promises and vat-crossing references are
  eventual; they carry only eventual-sends. Broken references carry
  neither. Promises may \emph{resolve} to near, far or
  broken. \emph{Partition} may break vat-crossing references}
\label{fig:refstates}
\end{figure}

We have added the possibility of a vat-crossing reference---a remote
promise or a far reference---getting broken by a partition. A
partition between a pair of vats eventually breaks all references that
cross between these vats, creating eventual common knowledge of the
loss of connection. A partition simultaneously breaks all references
crossing in a given direction between two vats. The sender of messages
that were still in transit cannot know which were actually received and
which were lost. Later messages will only be delivered by a reference
if all earlier messages sent on that same reference were already
delivered. This fail-stop FIFO delivery order relieves the sender from
needing to wait for earlier messages to be acknowledged before sending
later dependent messages.\footnote{
%
The message delivery order E enforces is stronger than FIFO and weaker
than Causal \cite{tribble:channels}, but FIFO is adequate for all
points we make in this paper.}

On our state-transition diagram (a Harel statechart), we see that
``near'' and ``broken'' are terminal states. Even after a partition
heals, all references broken by that partition stay broken.

In our listener example, if a partition separates the account's vat
from the spreadsheet's vat, the \abst{statusHolder}'s reference to the
spreadsheet's listener will eventually be broken with a
partition-exception. Of the \meth{statusChanged} messages sent by the
\abst{statusHolder}, this reference will deliver them reliably in FIFO
order until it fails. Once it fails to deliver a message, it will
never deliver any further messages and will eventually become
visibly broken.

An essential consequence of these semantics is that defensive
consistency is preserved across partition and reconnect.  A
defensively consistent program that makes no provisions for partition
remains defensively consistent. In the earlier \abst{statusHolder}
example, \meth{statusChanged} notifications sent to broken listener
references (e.g., broken because the connection to its subscriber vat was
severed) are harmlessly discarded.

\subsection{Handling Failure}

To explicitly manage failure of a reference, an object registers a
handler to be eventually notified when that reference becomes
broken. For the \abst{statusHolder} to clean up broken listener
references, it must register a handler on each one.
% 
\begin{alltt}
    to \dmeth{addListener}(\dvar{newListener}) \{
        myListeners.push(newListener)
        newListener <- statusChanged(myStatus)
        def \dobj{handler}(\dvar{}) \{ remove(myListeners, newListener) \}
        newListener <- \_\_whenBroken(handler)
    \}
\end{alltt}
%
The \meth{\_\_whenBroken} message is one of a handful of universally
understood messages that all objects respond to by default.\footnote{
%
In Java, the methods defined in \code{java.lang.Object} are similarly
universal.}
%
Of these, the following messages are for interacting with a reference
itself, as distinct from interacting only with the object designated
by a reference.
%
\begin{description}
\item[\code{\_\_whenBroken(\dvar{handler})}] When sent on a reference,
  this message registers its argument, \var{handler}, to be notified
  when this reference breaks.
\item[\code{\_\_whenMoreResolved(\dvar{handler})}] When sent on a
  reference, this message is normally used so that one can react when
  the reference is first resolved. We explain this in the later
  ``When-Catch'' section below.
\item[\code{\_\_reactToLostClient(\dvar{exception})}] When a
  vat-crossing reference breaks, it sends this message to its target
  object, to notify it that some of its clients may no longer be able
  to reach it.
\end{description}
%
Near references and local promises make no special case for these
messages---they merely deliver them to their targets. Objects by
default respond to a \meth{\_\_whenBroken} message by ignoring it,
because they are not broken. So, in our single-vat scenario, when all
these references are near, the additional code above has no effect. A
broken reference, on the other hand, responds by eventual-sending
a notification to the handler, as if by the following code:
%
\begin{alltt}
    to \dmeth{\_\_whenBroken}(\dvar{handler}) \{ handler <- run() \}
\end{alltt}
%
When a local promise gets broken, all its messages are forwarded to
the broken reference; when the \meth{\_\_whenBroken} message arrives, the
broken reference will notify the handler.

A vat-crossing reference notifies these handlers if it becomes broken,
whether by partition or resolution. In order to be able to send these
notifications during partition, a vat-crossing reference registers the
handler argument of a \meth{\_\_whenBroken} message at the tail end of
the reference, \emph{within the sending vat}. If the sending vat is
told that one of these references has resolved, it re-sends equivalent
\meth{\_\_whenBroken} messages to this resolution. If the sending vat
decides that a partition has occurred (perhaps because the internal
keep-alive timeout has been exceeded), it breaks all outgoing
references and notifies all registered handlers.

For all the reasons previously explained, the handler behavior built
into E's references only eventual-sends notifications to
handlers. Until the above handler reacts, the \abst{statusHolder} will
continue to harmlessly use the broken reference to the spreadsheet's
listener. Contingency concerns can thus be handled separately from normal
operation.

But what of the spreadsheet? We have ensured that it will receive
\meth{statusChanged} notifications in order, and that it will not miss
any in the middle of a sequence. But, during a partition, its display
may become arbitrarily stale. Technically, this introduces no new
consistency hazards because the data may be stale anyway due to
notification latencies. Nonetheless, the spreadsheet may wish to
provide a visual indication that the displayed value may now be more
stale than usual, since it is now out of contact with the
authoritative source. To make this convenient, when a reference is
broken by partition, it eventual-sends a \meth{\_\_reactToLostClient}
message to its target, notifying it that at least one of its clients
may no longer be able to send messages to it. By default, objects
ignore \meth{\_\_reactToLostClient} messages. The spreadsheet could
override the default behavior:
%
\begin{alltt}
    to \_\_reactToLostClient(\dvar{exception}) \{ {\it ...update display...} \}
\end{alltt}
%
Thus, when a vat-crossing reference is severed by partition,
notifications are eventually-sent to handlers at both ends of the
reference.  This explains how connectivity is safely severed by
partition and how objects on either side can react if they wish. 
Objects also need to regain connectivity following a partition. For
this purpose, we introduce \emph{offline capabilities}.

\subsection{Offline Capabilities}

An offline capability in E has two forms: a ``captp://...'' URI string
and an encapsulated \cls{SturdyRef} object. Both contain the same
information: the fingerprint of the public key of the vat hosting its
target object, a list of TCP/IP location hints to seed the search for
a vat that can authenticate against this fingerprint, and a so-called
\emph{swiss-number}, a large unguessable random number which the
hosting vat associates with the target \cite{tyler:yurl}. Like the
popular myth of how Swiss bank account numbers work, one demonstrates
knowledge of this secret to gain access to the object it designates.
Like an object reference, if you do not know an unguessable secret,
you can only come to know it if someone who knows it and can talk to
you chooses to tell it to you. An offline capability is a form of
``password capability''---it contains the cryptographic information
needed both to authenticate the target and to authorize access to the
target \cite{jed:dccs}.

Both forms of offline capability are pass-by-copy and can be passed
between vats even when the vat of the target object is inaccessible. Offline
capabilities do not directly convey messages to their target. To
establish or reestablish access to the target, one makes a new
reference from an offline capability. Doing so initiates a new attempt
to connect to the target vat and immediately returns a promise for
the resulting inter-vat reference. If the connection attempt fails,
this promise is eventually broken.

Typically, most inter-vat connectivity is only by references. When
these break, applications on either end should not try to recover the
detailed state of all the plans in progress between these
vats. Instead, they should typically spawn a new fresh structure from
the small number of offline capabilities from which this complex
structure was originally spawned. As part of this respawning process,
the two sides may need to explicitly reconcile in order to reestablish
distributed consistency.

In our listener example, the \abst{statusHolder} should not hold
offline capabilities to listeners and should not try to reconnect to
them. This would put the burden on the wrong party. A better design
would have a listener hold an offline capability to the
\abst{statusHolder}. The listener's \meth{\_\_reactToLostClient}
method would be enhanced to attempt to reconnect to the
\abst{statusHolder} and to resubscribe the listener on the promise for
the reconnected \abst{statusHolder}.

But perhaps the spreadsheet application originally encountered this
\abst{statusHolder} by navigating from an earlier object representing
a collection of accounts, creating and subscribing a spreadsheet cell
for each. While the vats were out of contact, not only may this
\abst{statusHolder} have changed, the collection may have changed so
that this \abst{statusHolder} is no longer relevant. In this case, a
better design would be for the spreadsheet to maintain an offline
capability only to the collection as a whole. When reconciling, it
should navigate afresh, in order to find the \abst{statusHolder}s to
which it should now subscribe.

The separation of references from offline capabilities encourages
programming patterns that separate reconciliation concerns from normal
operations.

\subsection{Persistence}

For an object that is designated only by references, the hosting vat
can tell when it is no longer reachable and can garbage-collect
it.\footnote{
%
E's distributed garbage collection protocol does not currently collect
unreachable inter-vat references cycles. See \cite{bejar:gc} for a
GC algorithm able to collect such cycles among mutually suspicious
machines.}
%
Once one makes an offline capability to a given object, its hosting vat
can no longer determine when it is unreachable. Instead, this vat must
retain the association between this object and its swiss-number until
its obligation to honor this offline capability expires.

The operations for making an offline capability provide three options
for ending this obligation: It can expire at a chosen future date,
giving the association a \emph{time-to-live}. It can expire when
explicitly cancelled, making the association \emph{revocable}. And it
can expire when the hosting vat crashes, making the association
\emph{transient}. Here, we examine only this last option. An
association which is not transient is \emph{durable}.

A vat can be either ephemeral or persistent. An ephemeral vat exists
only until it terminates or crashes; so for these, the last option
above is irrelevant. A persistent vat periodically \emph{checkpoints},
saving its persistent state to non-volatile storage. A vat checkpoints
only between turns when its stack is empty. A crash terminates a
vat-incarnation, rolling it back to its last checkpoint. Reviving the
vat from checkpoint creates a new incarnation of the same vat. A
persistent vat lives through a sequence of incarnations. With the
possibility of crash admitted into E's computational model, we can
allow programs to cause crashes, so they can preemptively terminate a
vat or abort an incarnation.

The persistent state of a vat is determined by traversal from
persistent roots. This state includes the vat's public/private key
pair, so later incarnations can authenticate. It also includes all
unexpired durable swiss-number associations and state reached by
traversal from there. As this traversal proceeds, when it reaches an
offline capability, the offline capability itself is saved but is not
traversed to its target. When the traversal reaches a vat-crossing
reference, a broken reference is saved instead and the reference is
again not traversed. Should this vat be revived from this checkpoint,
old vat-crossing references will be revived as broken references. A
crash partitions a vat from all others. Following a revival, only
offline capabilities in either direction enable it to become
reconnected.

\section{The When-Catch Expression}

The \meth{\_\_whenMoreResolved} message can be used to be register for notification
when a reference resolves. Typically this message is used indrectly
through the ``when-catch'' syntax. A when-catch expression takes a
promise, a ``when'' block to execute if the promise resolves to a
value, and a ``catch'' block to execute if the promise is broken.
This is illustrated by the following example.
%
\begin{alltt}
   def \dobj{asyncAnd}(\dvar{answers}) \{
       var \dvar{countDown} := answers.size()
       if (countDown == 0) \{ return true \}
       def [\dvar{result}, \dvar{resolver}] := Ref.promise()
       for \dvar{answer} in answers \{
           when (answer) -> \{
               if (answer) \{
                   countDown -= 1
                   if (countDown == 0) \{
                       resolver.resolve(true)
                   \}
               \} else \{
                   resolver.resolve(false)
               \}
           \} catch \dvar{exception} \{
               resolver.smash(exception)
           \}
       \}
       return result
   \}
\end{alltt}
%
The \var{asyncAnd} takes a list of promises for booleans. It
immediately returns a reference representing the conjunction, which
must eventually be true if all elements of the list become true, or
false or broken if any of them become false or broken.
Using when-catch, \var{asyncAnd} can test these as they become
available, so it can report a result as soon as it has enough
information.

If the list is empty, the conjunction is true right away. Otherwise,
\var{countDown} remembers how many true answers are needed before
\var{asyncAnd} can conclude that the conjunction is true. The
``when-catch'' expression is used to register a handler on each
reference in the list. The behavior of the handler is expressed in two
parts: the block after the ``\code{->}'' handles the normal case,
and the catch-clause handles the exceptional case. Once \var{answer}
resolves, if it is near or far, the normal-case code is run. If it is
broken, the catch-clause is run.  Here, if the normal case runs,
\var{answer} is expected to be a boolean. By using a ``when-catch'',
the \code{if} is postponed until \var{asyncAnd} has gathered enough information to
know which way it should branch.

Once \var{asyncAnd} registers all these handlers, it immediately
returns \var{result}, a promise for the conjunction of these
answers. If they all resolve to true, \var{asyncAnd} \emph{reveals}
that the result is true, i.e., it eventually resolves the
already-returned promise to true. If it is notified that any resolve to
false, \var{asyncAnd} reveals false immediately. If any resolve to
broken, \var{asyncAnd} reveals a reference broken by the same
exception. Asking a resolver to resolve an already-resolved promise
has no effect, so if one of the answers is false and another is broken,
the above \var{asyncAnd} code may reveal either false or broken,
depending on which handler happens to be notified first.

The following snippet illustrates using \var{asyncAnd} and when-catch
to combine independent validity checks in a toy application to resells
goods from a supplier.
%
\begin{alltt}
   def \dvar{allOk} := asyncAnd([inventory <- isAvailable(partNo),
                          creditBureau <- verifyCredit(buyerData),
                          shipper <- canDeliver(...)])
   when (allOk) -> \{
       if (allOk) \{
           def \dvar{receipt} := supplier <- buy(partNo, payment)
           when (receipt) -> \{
\end{alltt}
%
Promise-chaining postpones plans efficiently by data-flow; the
``when-catch'' postpones plans until the data needed for control-flow
is available.

\section{From Objects to Actors and Back Again}

Here we present a brief history of E's concurrency-control
architecture. In this section, the term ``we'' indicates that one or
both of this paper's first two authors participated in a project
involving other people. All implied credit should be understood as
shared with these others.

\related{Objects.} The nature of computation provided within a single
von Neumann machine is quite different than the nature of computation
provided by networks of such machines. Distributed programs must deal
with both. To reduce cases, it would seem attractive to create an
abstraction layer that can make these seem more similar. Distributed
Shared Memory systems try to make the network seem more like a von
Neumann machine. Object-oriented programming started by trying to make
a single computer seem more like a network.
%
\begin{quotation}
\ldots Smalltalk is a recursion on the notion of computer
itself. Instead of dividing ``computer stuff'' into things each less
strong than the whole---like data structures, procedures, and
functions which are the usual paraphernalia of programming
languages---each Smalltalk object is a recursion on the entire
possibilities of the computer. Thus its semantics are a bit like
having thousands and thousands of computers all hooked together by a
very fast network.
\begin{flushright}
---Alan Kay \cite{kay:smallhistory}
\end{flushright}
\end{quotation}
%
Smalltalk imported only the aspects of networks that made it easier to
program a single machine---its purpose was not to achieve network
transparency. Problems that could be avoided within a single
machine---like inherent asynchrony, large latencies, and partial
failures---were avoided. The sequential subset of E
has much in common with the early Smalltalk: Smalltalk's object
references are like E's near references and Smalltalk's message
passing is like E's immediate-call operator.

\related{Actors.} Inspired by the early Smalltalk, Hewitt created the
Actors paradigm \cite{hewitt:actors}, whose goals include full network
transparency within all the constraints imposed by decentralization
and mutual suspicion \cite{hewitt:challenge}. Although the stated
goals require the handling of partial failure, the actual Actors model
assumes this issue away and instead guarantees that all sent messages
are eventually delivered. The asynchronous-only subset of E is an
Actors language: Actors' references are like E's eventual references,
and Actors' message passing is much like E's eventual-send
operator. Actors provide both data-flow postponement of plans by
futures (like E's promises without pipelining or contagion) and
control-flow postponement by continuations (similar in effect to E's
when-catch).

The price of this uniformity is that all programs had to work
in the face of network problems. There was only one case to solve, but
it was the hard case.

\related{Vulcan.} Inspired by Shapiro and Takeuchi \cite{udi:objects},
the Vulcan project \cite{kahn:vulcan} merged aspects of Actors and
concurrent logic/constraint programming \cite{tr003,Saraswat93}. The
pleasant properties of concurrent logic variables (much like futures
or promises) taught us to emphasize data-flow postponement and
de-emphasize control-flow postponement.

Vulcan was built on a concurrent logic base, and inherited from 
it the so-called ``merge problem'' \cite{Shapiro:merge} absent from
pure Actors languages: Clients can only share access to a stateful
object by explicit pre-arrangement, so the equivalent of object
references were not usefully first-class. To address this problem, we
created the ``Channels'' abstraction, which also provides useful
ordering properties \cite{tribble:channels}.

\related{Joule.} The \sys{Joule} language \cite{tribble:joule} is a
capability-secure, massively-concurrent, distributed language that is
one of the primary precursors to E. Joule merges insights from the
\sys{Vulcan} project with the remaining virtues of Actors. \sys{Joule}
channels are similar to E's promises generalized to provide
multicasting. \sys{Joule} tanks are the unit of separate failure,
persistence, migration, and resource management, and inspired E vats.
E vats further define the unit of sequentiality; E's event-loop
approach achieves much of Joule's power with a more familiar and easy
to use computational model. Joule's resource management is based on
abstractions from KeyKOS \cite{hardy:keykos}. E vats do not yet
address this issue.

\related{Promise pipelining in Udanax Gold.} This was a pre-web
hypertext system with a rich interaction protocol between clients and
servers. To deal with network latencies, in the 1989 timeframe, we
independently reinvented an asymmetric form of promise pipelining as
part of our protocol design \cite{gold:promises}. This was the first
attempt to adapt Joule channels to an object-based client-server
environment (it did not support peer-to-peer). 

\related{Original-E.} The language now known as \sys{Original-E} was
the result of adding the concepts from Joule to the sequential,
capability-secure subset of Java. \sys{Original-E} was the first to
successfully mix sequential immediate-call programming with
asynchronous eventual-send programming. Original-E cryptographically
secured the Joule-like network extension---something that had been
planned for but not actually realized in prior systems.  Electric
Communities created \sys{Original-E}, and used it to build
\sys{Habitats}---a graphical, decentralized, secure, social virtual
reality system.

\related{From Original-E to E.} In \sys{Original-E}, the co-existence
of sequential and asynchronous programming was still rough. E brought
the invention of the distinct reference states and the transitions
among them explained in this paper. With these rules, E bridges the
gap between the network-as-metaphor view of the early Smalltalk and
the network-transparency ambitions of Actors. In E, the local case is
strictly easier than the network case, so the guarantees provided by
near references are a strict superset of the guarantees provided by
other reference states. When programming for known-local objects, a
programmer can do it the easy way. Otherwise, the programmer must
address the inherent problems of networks.  Once the programmer has
done so, the same code will painlessly also handle the local case
without requiring any further case analysis.

\section{Related Work}

\related{Promises and Batched Futures at MIT.} The promise
pipelining technique was first invented by Liskov and Shrira
\cite{liskov:promises}. These ideas were then significantly improved
by Bogle \cite{bogle:batched}. Like the \sys{Udanax Gold} system
mentioned above, these are asymmetric client-server systems. In other
ways, the techniques used in Bogle's protocol resembles quite closely
some of the techniques used in E's protocol.

\related{Group Membership.} There is an extensive body of work on
group membership systems \cite{birman:vsync,amir:thesis} and (broadly
speaking) similar systems such as Paxos \cite{lamport:paxos}. These
systems provide a different form of general-purpose framework for
dealing with partial failure: they support closer approximations of
common knowledge than does E, but at the price of weaker support for
defensive consistency and scalability. These frameworks better support
the tightly-coupled composition of separate plan-strands into a
virtual single overall plan. E's mechanisms better support the
loosely-coupled composition of networks of independent but cooperative
plans. 

For example, when a set of distributed components form an application
that provides a single logical service to all their collective
clients, and when multiple separated components may each change state
while out of contact with the others, we have a \emph{partition-aware
application} \cite{partition-aware,bancomat}, providing a form of
fault-tolerant replication. The clients of such an application see a
close approximation of a single stateful object that is highly
available under partition. Some mechanisms like group membership shine
at supporting this model under mutually reliant and even Byzantine
conditions \cite{castro:bft}.

E itself provides nothing comparable. The patterns of fault-tolerant
replication we have built to date are all forms of primary-copy
replication, with a single stationary authoritative host. E supports
these patterns quite well, and they compose well with simple E objects
that are unaware they are interacting with a replica. An area of
future research is to see how well partition-aware applications can be
programmed in E and how well they can compose with others.

\related{Croquet and TeaTime.} The \sys{Croquet} project has
many of the same goals as the \sys{Habitats} project referred to
above: to create a graphical, decentralized, secure, user-extensible,
social virtual reality system spread across mutually suspicious
machines. Regarding E, the salient differences are that \sys{Croquet}
is built on \sys{Smalltalk} extended onto the network by
\sys{TeaTime}, which is based on \sys{Namos} \cite{reed:namos} and
\sys{Paxos} \cite{lamport:paxos}, in order to replicate state among
multiple authoritative hosts. Unlike \sys{Habitats}, \sys{Croquet} is
user-extensible, but is not yet secure. It will be interesting to see
how they alter \sys{Paxos} to work between mutually suspicious
machines.

\subsection{Work influenced by E's concurrency control}

\related{The Web-Calculus} The \sys{Web-Calculus} \cite{tyler:webcalc}
brings to web URLs the following simultaneous properties:
%
\begin{itemize}
\item The cryptographic capability properties of E's offline
  capabilities---both authenticating the target and authorizing access
  to it.
\item Promise pipelining of eventually-POSTed requests with results.
\item The properties recommended by the REST model of web programming
  \cite{fielding:rest}. REST attributes the success of the web largely
  to certain loose-coupling properties of ``http://...''  URLs, which
  are well beyond the scope of this paper. See
  \cite{fielding:rest,tyler:webcalc} for more.
\end{itemize}
%
As a language-neutral protocol compatible and composable with existing
web standards, the \sys{Web-Calculus} is well-positioned 
to achieve widespread adoption. We expect to build a bridge between
E's references and \sys{Web-Calculus} URLs.

\related{Oz-E.} Like \sys{Vulcan}, the \sys{Oz} language
\cite{VanRoyHaridi} descends from both \sys{Actors} and concurrent
logic/constraint programming. Unlike these parents, \sys{Oz} supports
shared-state concurrency, though \sys{Oz} programming practice
discourages its use. \sys{Oz-E} \cite{oze} is a capability-based
successor to \sys{Oz} designed to support both local and distributed
defensive consistency. For the reasons explained in the ``Defensive
Correctness'' section above, \sys{Oz-E} suppresses \sys{Oz}'s
shared-state concurrency.

\related{Twisted Python.} This is a library and a set of
conventions for distributed programming in Python, based on E's model
of communicating event-loops, promise pipelining, and cryptographic
capability security \cite{twisted}.

\section{Discussion and Conclusions}

Electric Communities open-sourced E in 1998. Since then, a lively open
source community has continued development of E at
http://www.erights.org/. Seven companies and two universities have
used E---to teach secure and distributed programming, to rapidly
prototype distributed architectures, and to build several distributed
systems.

Despite these successful trials, we do not yet consider E ready for
production use---the current E implementation is a slow interpreter
written in Java.  Two compiler-based implementations are in progress:
Kevin Reid is building an E on Common Lisp \cite{reid:e-on-cl}, and
Dean Tribble is building an E on Squeak (an open-source
Smalltalk). Several of E's libraries, currently implemented in Java,
are being rewritten in E to help port E onto other language
platforms. Separately, Fred Spiessens continues to make progress on
formalizing the reasoning about \emph{authority} on which E's security
is based \cite{tgc}.

Throughout, our engineering premise is that lambda abstraction and
object programming, by their impressive plan coordination successes in
the small, have the seeds for coordinating plans in the large. As Alan
Kay has urged \cite{kay:ma}, our emphasis is less on the objects and
more on the interstitial fabric which connects them: the dynamic
reference graph carrying the messages by which their plans interact.

Encapsulation separates objects so their plans can avoid disrupting
each other's assumptions. Objects compose plans by message passing
while respecting each other's separation.  However, when client objects
request service from provider objects, their continued proper
functioning is often vulnerable to their provider's misbehavior. When
providers are also vulnerable to their clients, corruption is
potentially contagious over the reachable graph in both directions,
severely limiting the scale of systems we can compose.

Reduced vulnerability helps contain corruption. In this paper, we draw
attention to a specific composable standard of robustness: when a
provider is \emph{defensively consistent}, none of its clients can
corrupt it or cause it to give incorrect service to any of its
well-behaved clients, thus protecting its clients from each
other. When a system is composed of defensively consistent
abstractions, to a good approximation, corruption is contagious only
upstream. (Further vulnerability reduction beyond this standard is, of
course, valuable and often needed.)

Under shared-state concurrency---conventional multi-threading---we
have shown by example that defensive consistency is unreasonably
difficult. We have explained how an alternate concurrency-control
discipline, communicating event-loops, supports creating defensively
consistent objects in the face of concurrency and distribution. Our
enhanced reference graph consists of references in different states,
where their message delivery abilities depends on their state. Only
\emph{eventual references} convey messages between event-loops, and
deliver messages only in separately scheduled turns, providing
temporal separation of plans. \emph{Promises} pipeline messages
towards their likely destinations, compensating for
latency. \emph{Broken references} safely abstract partition, and
\emph{offline capabilities} abstract the ability to reconnect.

We have used small examples in this paper to illustrate principles with
which several projects have built large robust distributed systems.

\section{Acknowledgements}
For various helpful suggestions, we thank
Darius Bacon,
Dan Bornstein,
John Corbett,
Bill Frantz,
Ian Grigg,
Jim Hopwood,
Piotr Kaminski,
Alan Karp,
Matej Kosik,
Jon Leonard,
Kevin Reid,
Michael Sperber,
Fred Spiessens,
Terry Stanley,
Marc Stiegler,
Bill Tulloh,
Bryce ``Zooko'' Wilcox-O'Hearn,
Steve Witham,
and the e-lang and cap-talk communities.
We thank Terry Stanley for suggesting the listener pattern
and purchase-order examples.

We are especially grateful to Ka-Ping Yee and David Hopwood for a wide
variety of assistance. They reviewed numerous drafts, contributed
extensive and deep technical feedback, clarifying rephrasings, crisp
illustrations, and moral support.

\bibliography{common}
%
Links to on-line versions of many of these references are available at
http://www.erights.org/talks/promises/paper/references.html.

%\bibliographystyle{splncs}
\bibliographystyle{alpha}
\end{document}
